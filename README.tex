\documentclass[oneside,10pt,a4paper]{jsarticle}

\usepackage[dvips]{graphicx}
\usepackage[dvips]{graphicx,color}
\usepackage{amsmath}

\usepackage{commands/math_headers_ja}
\usepackage{commands/operators}
\usepackage{package/nard-tex-textbook-package/commands/headers_ja}

\title{nard-tex-math-package}
\author{Fujita Shu}

\begin{document}
  \maketitle

  \section{commands/math\_headers\_ja}

  \verb|commands/math_headers_ja| パッケージを読み込むと、以下のような数学の教科書の見出しのコマンドを使うことができる。

  \begin{itemize}
    \item \verb|\Axiom|
      \begin{itemize}
        \item 引数なし (\verb|\Axiom|)
          \begin{quote}
            \Axiom
          \end{quote}
        \item \verb|<>| で囲った引数あり (\verb|\Axiom<1>|)
          \begin{quote}
            \Axiom<1>
          \end{quote}
        \item \verb|[]| で囲った引数あり (\verb|\Axiom[選択公理]|)
          \begin{quote}
            \Axiom[選択公理]
          \end{quote}
        \item \verb|<>| で囲った引数、\verb|[]| で囲った引数あり (\verb|\Axiom<1>[選択公理]|)
          \begin{quote}
            \Axiom<1>[選択公理]
          \end{quote}
        %
      \end{itemize}
    %
    \item \verb|\Def|
      \begin{itemize}
        \item 引数なし (\verb|\Def|)
          \begin{quote}
            \Def
          \end{quote}
        \item \verb|<>| で囲った引数あり (\verb|\Def<2>|)
          \begin{quote}
            \Def<2>
          \end{quote}
        \item \verb|[]| で囲った引数あり (\verb|\Def[虚数]|)
          \begin{quote}
            \Def[虚数]
          \end{quote}
        \item \verb|<>| で囲った引数、\verb|[]| で囲った引数あり (\verb|\Def<2>[虚数]|)
          \begin{quote}
            \Def<2>[虚数]
          \end{quote}
        %
      \end{itemize}
    %
    \newpage
    %
    \item \verb|\Formula|
      \begin{itemize}
        \item 引数なし (\verb|\Formula|)
          \begin{quote}
            \Formula
          \end{quote}
        \item \verb|<>| で囲った引数あり (\verb|\Formula<3>|)
          \begin{quote}
            \Formula<3>
          \end{quote}
        \item \verb|[]| で囲った引数あり (\verb|\Formula[積和・和積の公式]|)
          \begin{quote}
            \Formula[積和・和積の公式]
          \end{quote}
        \item \verb|<>| で囲った引数、\verb|[]| で囲った引数あり (\verb|\Formula<3>[積和・和積の公式]|)
          \begin{quote}
            \Formula<3>[積和・和積の公式]
          \end{quote}
        %
      \end{itemize}
    %
    \item \verb|\Th|
      \begin{itemize}
        \item 引数なし (\verb|\Th|)
          \begin{quote}
            \Th
          \end{quote}
        \item \verb|<>| で囲った引数あり (\verb|\Th<4>|)
          \begin{quote}
            \Th<4>
          \end{quote}
        \item \verb|[]| で囲った引数あり (\verb|\Th[平均値の定理]|)
          \begin{quote}
            \Th[平均値の定理]
          \end{quote}
        \item \verb|<>| で囲った引数、\verb|[]| で囲った引数あり (\verb|\Th<4>[平均値の定理]|)
          \begin{quote}
            \Th<4>[平均値の定理]
          \end{quote}
        %
      \end{itemize}
    %
  \end{itemize}

  同様に、
  \begin{itemize}
    \item \verb|\Lem| - \Lem
    \item \verb|\Prop| - \Prop
    \item \verb|\Cor| - \Cor
    \item \verb|\Pf| - \Pf
  \end{itemize}
  も用いることができる。

  \newpage

  \section{commands/operators}

  \verb|commands/operators| パッケージを読み込むと、以下のような数学の関数などのコマンドを使うことができる。

  \subsection{三角関数}

  \begin{itemize}
    \item \verb|\cosec| - $ \cosec $
  \end{itemize}

  \subsection{逆三角関数}

  \begin{itemize}
    \item \verb|\Arcsin| - $ \Arcsin $
    \item \verb|\Arccos| - $ \Arccos $
    \item \verb|\Arctan| - $ \Arctan $
    \item \verb|\Arccsc| - $ \Arccsc $
    \item \verb|\Arcsec| - $ \Arcsec $
    \item \verb|\Arccot| - $ \Arccot $
  \end{itemize}

  \subsection{双曲線関数}

  \begin{itemize}
    \item \verb|\cosech| - $ \cosech $
    \item \verb|\csch| - $ \csch $
    \item \verb|\sech| - $ \sech $
  \end{itemize}

  \subsection{ベクトル解析}

  \begin{itemize}
    \item \verb|\grad| - $ \grad $
    \item \verb|\rot| - $ \rot $
    \item \verb|\vdiv| - $ \vdiv $
    \item \verb|\curl| - $ \curl $
  \end{itemize}

  \subsection{行列}

  \begin{itemize}
    \item \verb|\Tr| - $ \Tr $
    \item \verb|\SO| - $ \SO $
    \item \verb|\SU| - $ \SU $
  \end{itemize}

  \subsection{複素数}

  \begin{itemize}
    \item \verb|\re| - $ \re $
    \item \verb|\im| - $ \im $
  \end{itemize}

  \Remark
  \begin{quote}
    \verb|\Re|, \verb|\Im| はすでに定義されており、$ \Re $, $ \Im $ となる。
  \end{quote}

  \subsection{その他の演算子など}

  \begin{itemize}
    \item \verb|\sumprime| - $ \sumprime $
    \item \verb|\DefArrow| - $ \DefArrow $
  \end{itemize}
\end{document}
