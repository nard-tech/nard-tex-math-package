\documentclass[oneside,10pt,a4paper]{jsarticle}

\usepackage[dvips]{graphicx}
\usepackage[dvips]{graphicx,color}
\usepackage{amsmath}

\usepackage{commands/math_headers_ja}
\usepackage{commands/operators}
\usepackage{commands/utils}
\usepackage{commands/vector}
\usepackage{commands/matrix}
\usepackage{frac_style}

\usepackage{package/nard-tex-textbook-package/commands/headers_ja}
\usepackage{package/nard-tex-textbook-package/labelitem}

\title{nard-tex-math-package}
\author{Fujita Shu}

\begin{document}
  \maketitle

  \section{commands/math\_headers\_ja}

  \verb|commands/math_headers_ja| パッケージを読み込むと、以下のような数学の教科書の見出しのコマンドを使うことができる。

  \begin{itemize}
    \item \verb|\Axiom|
      \begin{itemize}
        \item 引数なし (\verb|\Axiom|)
          \begin{quote}
            \Axiom
          \end{quote}
        \item \verb|<>| で囲った引数あり (\verb|\Axiom<1>|)
          \begin{quote}
            \Axiom<1>
          \end{quote}
        \item \verb|[]| で囲った引数あり (\verb|\Axiom[選択公理]|)
          \begin{quote}
            \Axiom[選択公理]
          \end{quote}
        \item \verb|<>| で囲った引数、\verb|[]| で囲った引数あり (\verb|\Axiom<1>[選択公理]|)
          \begin{quote}
            \Axiom<1>[選択公理]
          \end{quote}
        %
      \end{itemize}
    %
    \item \verb|\Def|
      \begin{itemize}
        \item 引数なし (\verb|\Def|)
          \begin{quote}
            \Def
          \end{quote}
        \item \verb|<>| で囲った引数あり (\verb|\Def<2>|)
          \begin{quote}
            \Def<2>
          \end{quote}
        \item \verb|[]| で囲った引数あり (\verb|\Def[虚数]|)
          \begin{quote}
            \Def[虚数]
          \end{quote}
        \item \verb|<>| で囲った引数、\verb|[]| で囲った引数あり (\verb|\Def<2>[虚数]|)
          \begin{quote}
            \Def<2>[虚数]
          \end{quote}
        %
      \end{itemize}
    %
    \newpage
    %
    \item \verb|\Formula|
      \begin{itemize}
        \item 引数なし (\verb|\Formula|)
          \begin{quote}
            \Formula
          \end{quote}
        \item \verb|<>| で囲った引数あり (\verb|\Formula<3>|)
          \begin{quote}
            \Formula<3>
          \end{quote}
        \item \verb|[]| で囲った引数あり (\verb|\Formula[積和・和積の公式]|)
          \begin{quote}
            \Formula[積和・和積の公式]
          \end{quote}
        \item \verb|<>| で囲った引数、\verb|[]| で囲った引数あり (\verb|\Formula<3>[積和・和積の公式]|)
          \begin{quote}
            \Formula<3>[積和・和積の公式]
          \end{quote}
        %
      \end{itemize}
    %
    \item \verb|\Th|
      \begin{itemize}
        \item 引数なし (\verb|\Th|)
          \begin{quote}
            \Th
          \end{quote}
        \item \verb|<>| で囲った引数あり (\verb|\Th<4>|)
          \begin{quote}
            \Th<4>
          \end{quote}
        \item \verb|[]| で囲った引数あり (\verb|\Th[平均値の定理]|)
          \begin{quote}
            \Th[平均値の定理]
          \end{quote}
        \item \verb|<>| で囲った引数、\verb|[]| で囲った引数あり (\verb|\Th<4>[平均値の定理]|)
          \begin{quote}
            \Th<4>[平均値の定理]
          \end{quote}
        %
      \end{itemize}
    %
  \end{itemize}

  同様に、
  \begin{itemize}
    \item \verb|\Lem| - \Lem
    \item \verb|\Prop| - \Prop
    \item \verb|\Cor| - \Cor
    \item \verb|\Pf| - \Pf
  \end{itemize}
  も用いることができる。

  \newpage

  \section{commands/operators}

  \verb|commands/operators| パッケージを読み込むと、以下のような数学の関数などのコマンドを使うことができる。

  \subsection{三角関数}

  \begin{itemize}
    \item \verb|\cosec| - $ \cosec $
  \end{itemize}

  \subsection{逆三角関数}

  \begin{itemize}
    \item \verb|\Arcsin| - $ \Arcsin $
    \item \verb|\Arccos| - $ \Arccos $
    \item \verb|\Arctan| - $ \Arctan $
    \item \verb|\Arccsc| - $ \Arccsc $
    \item \verb|\Arcsec| - $ \Arcsec $
    \item \verb|\Arccot| - $ \Arccot $
  \end{itemize}

  \subsection{双曲線関数}

  \begin{itemize}
    \item \verb|\cosech| - $ \cosech $
    \item \verb|\csch| - $ \csch $
    \item \verb|\sech| - $ \sech $
  \end{itemize}

  \subsection{ベクトル解析}

  \begin{itemize}
    \item \verb|\grad| - $ \grad $
    \item \verb|\rot| - $ \rot $
    \item \verb|\vdiv| - $ \vdiv $
    \item \verb|\curl| - $ \curl $
  \end{itemize}

  \subsection{行列}

  \begin{itemize}
    \item \verb|\Trace| \footnote{\verb|\Tr| を別で使いたいのでコマンドをこのような名前にしている。(cf.  トレース \verb|\Tr| p.\pageref{Tr})} - $ \Trace $
    \item \verb|\SO| - $ \SO $
    \item \verb|\SU| - $ \SU $
  \end{itemize}

  \newpage

  \subsection{複素数}

  \begin{itemize}
    \item \verb|\re| - $ \re $
    \item \verb|\im| - $ \im $
  \end{itemize}

  \Remark
  \begin{quote}
    \verb|\Re|, \verb|\Im| はすでに定義されており、$ \Re $, $ \Im $ となる。
  \end{quote}

  \subsection{その他の演算子など}

  \begin{itemize}
    \item \verb|\sumprime| - $ \sumprime $
    \item \verb|\DefArrow| - $ \DefArrow $
  \end{itemize}

  \newpage

  \section{commands/utils}

  \verb|commands/utils| パッケージを読み込むと、以下のような数学に関する便利なコマンドを使うことができる。

  \begin{itemize}
    \item \verb|\Square| - 2乗
      \begin{quote}
        \Example
        \begin{itemize}
          \item \verb|\Square{x}| - $\Square{x}$
        \end{itemize}
      \end{quote}
    %
    \item \verb|\Inverse| - -1乗
      \begin{quote}
        \Example
        \begin{itemize}
          \item \verb|\Inverse{x}| - $\Inverse{x}$
        \end{itemize}
      \end{quote}
    %
    \item \verb|\Sqrt| - 平方根、$n$乗根
      \begin{quote}
        \Example
        \begin{itemize}
          \item \verb|\Sqrt{x}| - $\Sqrt{x}$
          \item \verb|\Sqrt[n]{x}| - $\Sqrt[n]{x}$
        \end{itemize}
      \end{quote}
    %
    \item \verb|\Abs| - 絶対値(少し広め)
      \begin{quote}
        \Example
        \begin{itemize}
          \item \verb|\Abs{x}| - $\Abs{x}$
        \end{itemize}
      \end{quote}
    %
    \item \verb|\Choose| - 二項係数
      \begin{quote}
        \Example
        \begin{itemize}
          \item \verb|\Choose{n}{k}| - $\Choose{n}{k}$
        \end{itemize}
      \end{quote}
    %
    \item \verb|\Conjugate| - 共役(少し広め)
      \begin{quote}
        \Example
        \begin{itemize}
          \item \verb|\Conjugate{A}| - $\Conjugate{A}$
        \end{itemize}
      \end{quote}
    %
    \item \verb|\Parentheses| - 括弧(少し広め)
      \begin{quote}
        \Example
        \begin{itemize}
          \item \verb|\Parentheses{x}| - $\Parentheses{x}$
        \end{itemize}
      \end{quote}
    %
    \item \verb|\Sequence| - 数列
      \begin{quote}
        \Example
        \begin{itemize}
          \item \verb|\Sequence{a_n}| - $\Sequence{a_n}$
        \end{itemize}
      \end{quote}
  \end{itemize}

  \newpage

  \subsection{黒板太字}

  \begin{itemize}
    \item 自然数 - \verb|\Natural| - $\Natural$
    \item 整数 - \verb|\Zahlen| - $\Zahlen$
    \item 有理数 - \verb|\Quotient| - $\Quotient$
    \item 実数 - \verb|\Real| - $\Real$
    \item 複素数 - \verb|\Complex| - $\Complex$
    \item ベクトル空間 - \verb|\V| - $\V$
    \item 体 - \verb|\K| - $\K$
  \end{itemize}

  \subsection{点と座標}

  \begin{itemize}
    \item \verb|\Point| - 点
      \begin{quote}
        \Example
        \begin{itemize}
          \item \verb|\Point{P}| - $\Point{P}$
          \item \verb|\Point{P}[a]| - $\Point{P}[a]$
          \item \verb|\Point{P}[1][2]| - $\Point{P}[1][2]$
          \item \verb|\Point{P}[1][2][3]| - $\Point{P}[1][2][3]$
        \end{itemize}
      \end{quote}
    %
    \item \verb|\Coordinate| - 座標
      \begin{quote}
        \Example
        \begin{itemize}
          \item 2次元 - \verb|\Coordinate[1][2]| - $\Coordinate[1][2]$
          \item 3次元 - \verb|\Coordinate[3][4][5]| - $\Coordinate[3][4][5]$
        \end{itemize}
      \end{quote}
    %
  \end{itemize}

  \subsection{関数}

  \begin{itemize}
    \item \verb|\Exp| - 指数関数
      \begin{quote}
        \Example
        \begin{itemize}
          \item \verb|\Exp{x}| - $\Exp{x}$
        \end{itemize}
      \end{quote}
    %
    \item \verb|\Function| - 一般の関数
      \begin{quote}
        3変数関数まで対応している。\\[6pt]
        \Example
        \begin{itemize}
          \item \verb|\Function{f}[x]| - $\Function{f}[x]$
          \item \verb|\Function{f}[x][y]| - $\Function{f}[x][y]$
          \item \verb|\Function{f}[x][y][z]| - $\Function{f}[x][y][z]$
        \end{itemize}
      \end{quote}
    %
  \end{itemize}

  \newpage

  \subsection{微分}

  \begin{itemize}
    \item \verb|\df|
      \begin{quote}
        \Example
        \begin{itemize}
          \item \verb|\df{}| - $\df{}$
          \item \verb|\df{x}| - $\df{x}$
          \item \verb|\df{f}| - $\df{f}$
          \item \verb|\df[2]{}| - $\df[2]{}$
          \item \verb|\df[2]{x}| - $\df[2]{x}$(\verb|\df{x^2}| と異なることに注意)
          \item \verb|\df{x^2}| - $\df[2]{x^2}$(\verb|\df[2]{x}| と異なることに注意)
          \item \verb|\df[n]{}| - $\df[n]{}$
          \item \verb|\df[n]{f}| - $\df[n]{f}$
          \item \verb|\int_{0}^{1} x^{2} \df{x}| - $\displaystyle \int_{0}^{1} x^{2} \df{x}$
        \end{itemize}
      \end{quote}
    %
    \item \verb|\Df|
      \begin{quote}
        \Example
        \begin{itemize}
          \item \verb|\Df{}| - $\displaystyle \Df{}{x}$ \\[-6pt]
          \item \verb|\Df[2]{}{x}| - $\displaystyle \Df[2]{}{x}$ \\[-6pt]
          \item \verb|\Df[n]{}{x}| - $\displaystyle \Df[n]{}{x}$ \\[-6pt]
          \item \verb|\Df{f}{x}| - $\displaystyle \Df{f}{x}$ \\[-6pt]
          \item \verb|\Df[2]{f}{x}| - $\displaystyle \Df[2]{f}{x}$ \\[-6pt]
          \item \verb|\Df[n]{f}{x}| - $\displaystyle \Df[n]{f}{x}$ \\[-6pt]
          \item 他のコマンドとの組み合わせ
            \begin{itemize}
              \item \verb|\Df{}{x} \Function{f}[x]| - $\displaystyle \Df{}{x} \Function{f}[x]$ \\[-6pt]
              \item \verb|\Df[2]{}{x} \Function{f}[x]| - $\displaystyle \Df[2]{}{x} \Function{f}[x]$ \\[-6pt]
              \item \verb|\Df{\Function{f}[x]}{x}| - $\displaystyle \Df{\Function{f}[x]}{x}$ \\[-6pt]
              \item \verb|\Function{\Df{f}{x}}[x]| - $\displaystyle \Function{\Df{f}{x}}[x]$
            \end{itemize}
          %
        \end{itemize}
      \end{quote}
    %
  \end{itemize}

  \newpage

  \subsection{偏微分}

  \begin{itemize}
    \item \verb|\pdf|
      \begin{quote}
        \Example
        \begin{itemize}
          \item \verb|\pdf{}| - $\pdf{}$
          \item \verb|\pdf{x}| - $\pdf{x}$
          \item \verb|\pdf{f}| - $\pdf{f}$
          \item \verb|\pdf[2]{}| - $\pdf[2]{}$
          \item \verb|\pdf[2]{x}| - $\pdf[2]{x}$(\verb|\pdf{x^2}| と異なることに注意)
          \item \verb|\pdf{x^2}| - $\pdf[2]{x^2}$(\verb|\pdf[2]{x}| と異なることに注意)
          \item \verb|\pdf[n]{}| - $\pdf[n]{}$
          \item \verb|\pdf[n]{f}| - $\pdf[n]{f}$
        \end{itemize}
      \end{quote}
    %
    \item \verb|\Pdf| - 分数型表記 - 1階偏微分...n階偏微分に対応
      \begin{quote}
        \Example
        \begin{itemize}
          \item \verb|\Pdf{}| - $\displaystyle \Pdf{}{x}$ \\[-6pt]
          \item \verb|\Pdf[2]{}{x}| - $\displaystyle \Pdf[2]{}{x}$ \\[-6pt]
          \item \verb|\Pdf[n]{}{x}| - $\displaystyle \Pdf[n]{}{x}$ \\[-6pt]
          \item \verb|\Pdf{f}{x}| - $\displaystyle \Pdf{f}{x}$ \\[-6pt]
          \item \verb|\Pdf[2]{f}{x}| - $\displaystyle \Pdf[2]{f}{x}$ \\[-6pt]
          \item \verb|\Pdf[n]{f}{x}| - $\displaystyle \Pdf[n]{f}{x}$ \\[-6pt]
          \item 他のコマンドとの組み合わせ
            \begin{itemize}
              \item \verb|\Pdf{}{x} \Function{f}[x][y]| - $\displaystyle \Pdf{}{x} \Function{f}[x][y]$ \\[-6pt]
              \item \verb|\Pdf[2]{}{x} \Function{f}[x][y]| - $\displaystyle \Pdf[2]{}{x} \Function{f}[x][y]$ \\[-6pt]
              \item \verb|\Pdf{\Function{f}[x][y]}{x}| - $\displaystyle \Pdf{\Function{f}[x][y]}{x}$ \\[-6pt]
              \item \verb|\Function{\Pdf{f}{x}}[x][y]| - $\displaystyle \Function{\Pdf{f}{x}}[x][y]$
            \end{itemize}
          %
        \end{itemize}
        \quad \\
      \end{quote}
    %
    \item \verb|\Pdft| - 1階偏微分 熱力学でよく使う表記
      \begin{quote}
        \Example
        \begin{itemize}
          \item \verb|\Pdft{z}{x}{y}| - $\displaystyle \Pdft{z}{x}{y}$ \\[-6pt]
          \item \verb|\Pdft{p}{S}{V}| - $\displaystyle \Pdft{p}{S}{V}$
        \end{itemize}
        \quad \\
      \end{quote}
    %
    \item \verb|\Ppdf| - 分数型表記 - 2階偏微分(異なる変数で2回偏微分する場合)
      \begin{quote}
        \Example
        \begin{itemize}
          \item \verb|\Ppdf{}{x}{y} \Function{f}[x][y]| - $\displaystyle \Ppdf{}{x}{y} \Function{f}[x][y]$ \\[-6pt]
          \item \verb|\Ppdf{\Function{f}[x][y]}{x}{y}| - $\displaystyle \Ppdf{\Function{f}[x][y]}{x}{y}$
        \end{itemize}
        \quad \\
      \end{quote}
  \end{itemize}

  \newpage

  \section{commands/vector}

  \verb|commands/vector| パッケージを読み込むと、以下のようなベクトルに関するコマンドを使うことができる。

  \begin{itemize}
    \item \verb|\VectorArrow| - 矢印付きベクトル
      \begin{quote}
        \Example
        \begin{itemize}
          \item \verb|\VectorArrow{a}| - $\VectorArrow{a}$
          \item \verb|\VectorArrow{a} + \VectorArrow{b}| - $\VectorArrow{a} + \VectorArrow{b}$
        \end{itemize}
        \,
      \end{quote}
    %
    \item \verb|\Norm| - ノルム
      \begin{quote}
        \Example
        \begin{itemize}
          \item \verb|\Norm{a}| - $\Norm{a}$
        \end{itemize}
        \,
      \end{quote}
    %
    \item \verb|\InPr| - 内積
      \begin{quote}
        \Example
        \begin{itemize}
          \item \verb|\InPr{\mathbf{a}}{\mathbf{b}}| - $\InPr{\mathbf{a}}{\mathbf{b}}$
        \end{itemize}
        \,
      \end{quote}
    %
    \item \verb|\ColumnVectorDouble| - 列ベクトル(2行)
      \begin{quote}
        \Example
        \begin{itemize}
          \item \verb|\ColumnVectorDouble{1}{2}| - $\ColumnVectorDouble{1}{2}$
        \end{itemize}
        \,
      \end{quote}
    %
    \item \verb|\ColumnVectorTriple| - 列ベクトル(3行)
      \begin{quote}
        \Example
        \begin{itemize}
          \item \verb|\ColumnVectorTriple{1}{2}{3}| - $\ColumnVectorTriple{1}{2}{3}$
        \end{itemize}
        \,
      \end{quote}
    %
    \item \verb|\RowVectorDouble| - 行ベクトル(2列)
      \begin{quote}
        \Example
        \begin{itemize}
          \item \verb|\RowVectorDouble{1}{2}| - $\RowVectorDouble{1}{2}$
        \end{itemize}
        \,
      \end{quote}
    %
    \item \verb|\RowVectorTriple| - 行ベクトル(3列)
      \begin{quote}
        \Example
        \begin{itemize}
          \item \verb|\RowVectorTriple{1}{2}{3}| - $\RowVectorTriple{1}{2}{3}$
        \end{itemize}
        \,
      \end{quote}
    %
    \item \verb|\ScalarTripleProduct| - スカラー3重積
      \begin{quote}
        \Example
        \begin{itemize}
          \item \verb|\ScalarTripleProduct{\VectorArrow{a}}{\VectorArrow{b}}{\VectorArrow{c}}| - \\ $\ScalarTripleProduct{\VectorArrow{a}}{\VectorArrow{b}}{\VectorArrow{c}}$
        \end{itemize}
        % \,
      \end{quote}
    %
  \end{itemize}

  \section{commands/matrix}

  \verb|commands/matrix| パッケージを読み込むと、以下のような行列に関するコマンドを使うことができる。

  \begin{itemize}
    \item \verb|\hsymb| - 大きな O など(cf. $n$ 次単位行列 \verb|\NthUnitMatrix| p.\pageref{NthUnitMatrix})
      \begin{quote}
        \Example
        \begin{itemize}
          \item \verb|\hsymb{O}| - $\hsymb{O}$
        \end{itemize}
      \end{quote}
    %
    \item \verb|\MatrixSize| - 「$ m \times n $行列」
      \begin{quote}
        \Remark
        \begin{quote}
          テキストモードで使用する。
        \end{quote}
        \Example
        \begin{itemize}
          \item \verb|\MatrixSize{2}{3}| - \MatrixSize{2}{3}
        \end{itemize}
      \end{quote}
    %
    \item \verb|\Pmatrix| - 丸括弧付き行列
      \begin{quote}
        \Example
        \begin{itemize}
          \item \verb|\Pmatrix{\begin{array}{cc} 1 & 2 \\ 3 & 4 \end{array}}| - $
              \Pmatrix{
                \begin{array}{cc}
                  1 & 2 \\
                  3 & 4
                \end{array}
              }
            $
          %
        \end{itemize}
      \end{quote}
    %
    \item \verb|\MatrixElement| - 行列の成分
      \begin{quote}
        \Example
        \begin{itemize}
          \item \verb|$\MatrixElement{i}{j}$ 成分| - $\MatrixElement{i}{j}$ 成分
          \item \verb|A = {\left( a_{ij} \right)}_{\MatrixElement{i}{j}}| - $A = {\left( a_{ij} \right)}_{\MatrixElement{i}{j}}$
        \end{itemize}
      \end{quote}
    %
    \item \verb|\Det| - 行列式
      \begin{quote}
        \Example
        \begin{itemize}
          \item \verb|\Det{A}| - $\Det{A}$
          \item \verb|\Det{\Pmatrix{\begin{array}{cc} 1 & 2 \\ 3 & 4 \end{array}}}| - $\Det{\Pmatrix{\begin{array}{cc} 1 & 2 \\ 3 & 4 \end{array}}}$
        \end{itemize}
      \end{quote}
    %
    \item \verb|\Tr| - トレース \label{Tr}
      \begin{quote}
        \Example
        \begin{itemize}
          \item \verb|\Tr{A}| - $\Tr{A}$
          \item \verb|\Tr{\Pmatrix{\begin{array}{cc} 1 & 2 \\ 3 & 4 \end{array}}}| - $\Tr{\Pmatrix{\begin{array}{cc} 1 & 2 \\ 3 & 4 \end{array}}}$
        \end{itemize}
      \end{quote}
    %
    \item \verb|\TransposedMatrix| - 転置行列
      \begin{quote}
        \Example
        \begin{itemize}
          \item \verb|\TransposedMatrix{A}| - $\TransposedMatrix{A}$
          \item \verb|\TransposedMatrix{\Pmatrix{\begin{array}{cc} 1 & 2 \\ 3 & 4 \end{array}}}| \\ $\TransposedMatrix{\Pmatrix{\begin{array}{cc} 1 & 2 \\ 3 & 4 \end{array}}}$
        \end{itemize}
      \end{quote}
    %
    \newpage
    %
    \item \verb|\AdjointMatrix| - 随伴行列
      \begin{quote}
        \Example
        \begin{itemize}
          \item \verb|\AdjointMatrix{A}| - $\AdjointMatrix{A}$
          \item \verb|\AdjointMatrix{\Pmatrix{\begin{array}{cc} 1 & 2 \\ 3 & 4 \end{array}}}| - \\ $\AdjointMatrix{\Pmatrix{\begin{array}{cc} 1 & 2 \\ 3 & 4 \end{array}}}$
        \end{itemize}
      \end{quote}
    %
  \end{itemize}

  \newpage

  \subsection{2次正方行列}

  \begin{itemize}
    \item \verb|\SecondSquareMatrix|, \verb|\SecondSquareMatrixR| - 2次正方行列
      \begin{quote}
        \Example
        \begin{itemize}
          \item \verb|\SecondSquareMatrix{1}{2}{3}{4}| - $\SecondSquareMatrix{1}{2}{3}{4}$
          \item \verb|\SecondSquareMatrix{1}{2}{-3}{-4}| - $\SecondSquareMatrix{1}{2}{-3}{-4}$
          \item \verb|\SecondSquareMatrixR{1}{2}{-3}{-4}| - $\SecondSquareMatrixR{1}{2}{-3}{-4}$
          \item \verb|\SecondSquareMatrix[rr]{1}{2}{-3}{-4}| - $\SecondSquareMatrix[rr]{1}{2}{-3}{-4}$
        \end{itemize}
      \end{quote}
    %
    \item \verb|\SecondUnitMatrix| - 2次単位行列
      \begin{quote}
        \Example
        \begin{itemize}
          \item \verb|\SecondUnitMatrix| - $\SecondUnitMatrix$
        \end{itemize}
      \end{quote}
    %
    \item \verb|\SecondSquareMatrixGeneral| - 2次正方行列(標準)
      \begin{quote}
        \Example
        \begin{itemize}
          \item \verb|\SecondSquareMatrixGeneral{a}| - $\SecondSquareMatrixGeneral{a}$
        \end{itemize}
      \end{quote}
    %
    \item \verb|\SecondSquareMatrixGeneralTransposed| - 2次正方行列(標準・転置)
      \begin{quote}
        \Example
        \begin{itemize}
          \item \verb|\SecondSquareMatrixGeneralTransposed{a}| - $\SecondSquareMatrixGeneralTransposed{a}$
        \end{itemize}
      \end{quote}
    %
    \item \verb|\SecondSquareMatrixGeneralSum| - 2次正方行列(標準・和)
      \begin{quote}
        \Example
        \begin{itemize}
          \item \verb|\SecondSquareMatrixGeneralSum{a}{b}| - $\SecondSquareMatrixGeneralSum{a}{b}$
        \end{itemize}
      \end{quote}
    %
  \end{itemize}

  \newpage

  \subsection{3次正方行列}

  \begin{itemize}
    \item \verb|\ThirdSquareMatrix|, \verb|\ThirdSquareMatrixR| - 3次正方行列
      \begin{quote}
        \Example
        \begin{itemize}
          \item \verb|\ThirdSquareMatrix{1}{2}{3}{4}{5}{6}{7}{8}{9}| - $\ThirdSquareMatrix{1}{2}{3}{4}{5}{6}{7}{8}{9}$
          \item \verb|\ThirdSquareMatrix{1}{-2}{-3}{-4}{5}{6}{-7}{-8}{9}| - $\ThirdSquareMatrix{1}{-2}{-3}{-4}{5}{6}{-7}{-8}{9}$
          \item \verb|\ThirdSquareMatrixR{1}{-2}{-3}{-4}{5}{6}{-7}{-8}{9}| - $\ThirdSquareMatrixR{1}{-2}{-3}{-4}{5}{6}{-7}{-8}{9}$
        \end{itemize}
        \Remark
        \begin{quote}
          引数の個数の都合上、\verb|\ThirdSquareMatrix[rrr]| は使えない。
        \end{quote}
      \end{quote}
    %
    \item \verb|\ThirdUnitMatrix| - 3次単位行列
      \begin{quote}
        \Example
        \begin{itemize}
          \item \verb|\ThirdUnitMatrix| - $\ThirdUnitMatrix$
        \end{itemize}
      \end{quote}
    %
    \item \verb|\ThirdSquareMatrixGeneral| - 3次正方行列(標準)
      \begin{quote}
        \Example
        \begin{itemize}
          \item \verb|\ThirdSquareMatrixGeneral{a}| - $\ThirdSquareMatrixGeneral{a}$
        \end{itemize}
      \end{quote}
    %
    \item \verb|\ThirdSquareMatrixGeneralTransposed| - 3次正方行列(標準・転置)
      \begin{quote}
        \Example
        \begin{itemize}
          \item \verb|\ThirdSquareMatrixGeneralTransposed{a}| - $\ThirdSquareMatrixGeneralTransposed{a}$
        \end{itemize}
      \end{quote}
    %
    \item \verb|\ThirdSquareMatrixGeneralSum| - 3次正方行列(標準・和)
      \begin{quote}
        \Example
        \begin{itemize}
          \item \verb|\ThirdSquareMatrixGeneralSum{a}{b}| - $\ThirdSquareMatrixGeneralSum{a}{b}$
        \end{itemize}
      \end{quote}
    %
    \newpage
    %
    \item \verb|\ThirdDeterminantGeneral| - 3次正方行列 行列式(標準)
      \begin{quote}
        \Example
        \begin{itemize}
          \item \verb|\ThirdDeterminantGeneral{a}| - \\ $\ThirdDeterminantGeneral{a}$
        \end{itemize}
      \end{quote}
    %
    \item 3次正方行列 小行列式
      \begin{quote}
        \Example
        \begin{itemize}
          \item \verb|\ThirdMinorDeterminantIi{a}| - $\ThirdMinorDeterminantIi{a}$
          \item \verb|\ThirdMinorDeterminantIii{a}| - $\ThirdMinorDeterminantIii{a}$
          \item \verb|\ThirdMinorDeterminantIiii{a}| - $\ThirdMinorDeterminantIiii{a}$
          \item \verb|\ThirdMinorDeterminantIIi{a}| - $\ThirdMinorDeterminantIIi{a}$
          \item \verb|\ThirdMinorDeterminantIIii{a}| - $\ThirdMinorDeterminantIIii{a}$
          \item \verb|\ThirdMinorDeterminantIIiii{a}| - $\ThirdMinorDeterminantIIiii{a}$
          \item \verb|\ThirdMinorDeterminantIIIi{a}| - $\ThirdMinorDeterminantIIIi{a}$
          \item \verb|\ThirdMinorDeterminantIIIii{a}| - $\ThirdMinorDeterminantIIIii{a}$
          \item \verb|\ThirdMinorDeterminantIIIiii{a}| - $\ThirdMinorDeterminantIIIiii{a}$
        \end{itemize}
      \end{quote}
    %
  \end{itemize}

  \newpage

  \subsection{$n$ 次正方行列}

  \begin{itemize}
    \item \verb|\NthSquareMatrix|, \verb|\NthSquareMatrixR| - $n$ 次正方行列
      \begin{quote}
        \Example
        \begin{itemize}
          \item \verb|\NthSquareMatrix{1}{2}{3}{4}{5}{6}{7}{8}{9}| - $\NthSquareMatrix{1}{2}{3}{4}{5}{6}{7}{8}{9}$
          \item \verb|\NthSquareMatrix{1}{-2}{-3}{-4}{5}{-6}{-7}{8}{-9}| - $\NthSquareMatrix{1}{-2}{-3}{-4}{5}{-6}{-7}{8}{-9}$
          \item \verb|\NthSquareMatrixR{1}{-2}{-3}{-4}{5}{-6}{-7}{8}{-9}| - $\NthSquareMatrixR{1}{-2}{-3}{-4}{5}{-6}{-7}{8}{-9}$
        \end{itemize}
      \end{quote}
    %
    \item \verb|\NthUnitMatrix| - $n$ 次単位行列 \label{NthUnitMatrix}
      \begin{quote}
        \Example
        \begin{itemize}
          \item \verb|\NthUnitMatrix| - $\NthUnitMatrix$
        \end{itemize}
      \end{quote}

    \item \verb|\NthUnitMatrix| - $n$ 次正方行列(標準)
      \begin{quote}
        \Example
        \begin{itemize}
          \item \verb|\NthSquareMatrixGeneral{a}| - $\NthSquareMatrixGeneral{a}$
        \end{itemize}
      \end{quote}
    %
    \item \verb|\NthSquareMatrixGeneralTransposed| - $n$ 次正方行列(標準・転置)
      \begin{quote}
        \Example
        \begin{itemize}
          \item \verb|\NthSquareMatrixGeneralTransposed{a}| - $\NthSquareMatrixGeneralTransposed{a}$
        \end{itemize}
      \end{quote}
    %
    \item \verb|\NthSquareMatrixGeneralSum| - $n$ 次正方行列(標準・和)
      \begin{quote}
        \Example
        \begin{itemize}
          \item \verb|\NthSquareMatrixGeneralSum{a}{b}| - $\NthSquareMatrixGeneralSum{a}{b}$
        \end{itemize}
      \end{quote}
    %
  \end{itemize}

  \subsection{Pauli 行列}

  \begin{itemize}
    \item \verb|\PauliMatrixI|, \verb|\PauliMatrixII|, \verb|\PauliMatrixIII| - Pauli(パウリ)行列
      \begin{quote}
        \Example
        \begin{itemize}
          \item \verb|\PauliMatrixI| - $\PauliMatrixI$
          \item \verb|\PauliMatrixII| - $\PauliMatrixII$
          \item \verb|\PauliMatrixIII| - $\PauliMatrixIII$
        \end{itemize}
      \end{quote}
    %
  \end{itemize}
\end{document}
